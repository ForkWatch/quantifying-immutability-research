\documentclass{article}


\usepackage{arxiv}

\usepackage[utf8]{inputenc} % allow utf-8 input
\usepackage[T1]{fontenc}    % use 8-bit T1 fonts
\usepackage{hyperref}       % hyperlinks
\usepackage{url}            % simple URL typesetting
\usepackage{booktabs}       % professional-quality tables
\usepackage{amsfonts}       % blackboard math symbols
\usepackage{nicefrac}       % compact symbols for 1/2, etc.
\usepackage{microtype}      % microtypography
\usepackage{lipsum}

\title{
        Quantifying Immutability in Blockchains\\
        \large The Finney Ratio and Szabo Score
}
\subtitle{}


\author{
        Yaz Khoury\thanks{Website is https://www.yazkhoury.com. Other email is: yaz.khoury@gmail.com} \\
        Director of Developer Relations\\
        Ethereum Classic Cooperative\\
        \texttt{yaz@etccooperative.org} \\
}


\begin{document}
\maketitle

\begin{abstract}
Immutability is a term used a lot in blockchain and cryptocurrency development when attempting to analyze the economic interpretation of a network. Designs like the 21 million Bitcoin supply in BTC or incidents
like the DAO fork bailout that split the Ethereum community into ETH and ETC are major examples of immutability values determining the ethos of a chain. This research attempts at categorizing different features
of blockchain immutability and attempts to analyze them in three different networks: Bitcoin (BTC), Ethereum (ETH), and Ethereum Classic (ETC). Further in the paper, we introduce the Finney Ratio to help understand
blockchain immutability over time. We then introduce the Szabo Score in order to score each network based on its overall immutability.
\end{abstract}

\keywords{Immutability \and Blockchain \and Finney Ratio \and Szabo Score}

\end{document}
